
% Functionality
Modern key-value stores, including HBase, follow the design of Google's Bigtable~\cite{Chang2008}.
This section sketches their common design principles and terminology. 

\paragraph{Data Layout}
KV-stores hold data items (referred to as \emph{rows}) identified by unique 
row keys. Each row can consist of multiple field (\emph{columns}) identified by unique 
column keys. Co-accessed columns (typically used by the same application) can be 
aggregated into  \emph{column families} to optimize access. The data is multi-versioned, 
i.e., multiple versions of the same row key can exist, identified by unique {\em timestamps}. 
The smallest unit of data, named {\em cell}, is defined by a combination of a row key, a
column key, and a timestamp.

The basic KV-store API includes \emph{put} (point update of one or more cells, by row key), 
\emph{get} (point query of one or more cells, by row key), and \emph{scan} (range query of one 
 of one or more cells, of all keys between an ordered pair of begin and end keys). 

\begin{figure}[tb]
\center
\includegraphics[width=\columnwidth, trim={0 2cm 0 2cm}, clip]{LSM} 
\caption{A log-sequence-merge LSM tree consists of a small memory store ({\em MemStore}, in HBase) 
and a large disk store (collection of {\em HFile}'s). Put operations update the MemStore. The latter is 
double-buffered: a flush freezes the mutable buffer, and creates a new one. 
The immutable buffer is written to disk in the background.}
\label{fig:LSM}
\end{figure}

\paragraph{Data Management}
KV-stores achieve scalability by sharding tables into range partitions 
by row key. A shard is called {\em region\/} in HBase (\emph{tablet} in Bigtable~\cite{Chang2008}). 
A region is a collection of \emph{stores}, each associated with a column family. 
Each store is backed by a collection of files sorted by row key, called \emph{HFile}'s in HBase 
(\emph{sst} files in Bigtable). HBase is layered on top of Hadoop Filesystem (HDFS), which 
replicates data for availability. 

Data access in HBase is provided through {\em region servers} (analogous to {\em tablet servers}
in Bigtable). Each region server controls multiple stores (tens to hundreds in production settings). 
For locality of access, the HBase management plane tries to lay out the HFile's in a way that their 
primary replicas are colocated with the region servers that control them. Production HBase clusters
consist of hundreds to thousands of nodes. 

\paragraph{LSM trees}
Each store is organized as log-sequence-merge (LSM) tree, which collects writes in a memory store 
(\emph{MemStore} in HBase), and periodically flushes the memory into a disk store, as illustrated in 
Figure~\ref{fig:LSM}. Each flush creates a new immutable HFile, ordered by row key for query efficiency. 
HDFS is optimized for very large files (its default block size is $64$MB), hence flushes are relatively infrequent. 

To allow puts to proceed in parallel with I/O, MemStore employs double buffering. 
It maintains a dynamic \emph{active} buffer absorbing puts, and a static \emph{frozen}
buffer that holds the previous version of the active buffer. The latter is written to the 
filesystem in the background. Both buffers are ordered by key, as are the HFile's.  

% LSM puts and flushes
Put operations are directed to MemStore. A flush occurs either when the active buffer 
exceeds the region size limit ($128$ MB, by default), or when the overall footprint of all MemStore's
in the region server exceeds the global size limit ($40\%$ of the heap, by default). 
Flush first freezes the current active buffer, making it immutable, and creates a new empty one.
It then replaces the reference to the active buffer, and proceeds to write the immutable buffer 
as a new HFile. 

% WAL
In order to guarantee durability of writes between flushes, updates are first written to 
the write-ahead log (WAL). HBase implements the logical WAL as collection of one or more physical 
logs per region server, called \emph{HLog}'s. One HLog maps to multiple log files on HDFS. 
As new data gets flushed to HFile's, the old log files become redundant; the system collects 
them in the background. On the other hand, HLog growth beyond boundary may trigger flushes. 
Since the logged data is only required at recovery time and is only scanned sequentially, the HLog's 
may be stored on slower devices than the HFile's in production (e.g., HDD's vs SSD's),
through the HDFS heterogeneous storage configuration. 

To keep the number of HFile's per store bounded, \emph{compactions} merge multiple files 
into one, while eliminating redundant (overwritten) versions. From time to time, \emph{major} 
compactions merge all files in one. These compactions also eliminate the special versions 
({\em tombstones}) that indicate row deletions. Major compactions incur huge performance
impact to the concurrent operations; they are carefully scheduled in production. 

% LSM gets
The get operation searches for the key in parallel in the MemStore (both the active and the 
frozen buffers) and in the HFile's. The search in the HFile's is sped up through Bloom 
filters~\cite{Chang2008}, which eliminate most redundant disk reads. The system 
uses a large RAM cache for popular HFile blocks (by default, $40\%$ of the heap).

\paragraph{Memory organization}
The MemStore active buffer is traditionally implemented as a dynamic index over a collection cells.  
The HBase implementation uses the standard Java concurrent skiplist\footnote{\small{\url{https://docs.oracle.com/javase/7/docs/api/java/util/concurrent/ConcurrentSkipListSet.html}}}).
Data is multi-versioned, i.e., every put creates a new immutable version of the row it is applied to, 
consisting of one or more cells. 

This implementation suffers from two drawbacks. First, the use of a big dynamic data structure entails 
the abundance of auxiliary small objects and references, which inflate the in-memory index. 
The overhead is most significant when the managed objects
are small, i.e., the metadata-to-data ratio is big~\cite{Wu2015}. Second, the versioning mechanism 
makes no attempt to eliminate redundancies prior to flush, i.e., the MemStore size  steadily grows, 
independently of the workload. This is especially wasteful for heavy-tailed distributions that are 
prevalent in production workloads~\cite{Devineni:2015}. 

The \sys\/ algorithm takes care of precisely these two problems, in order to boost the 
LSM tree performance and reduce the storage wearout. Section~\ref{sec:accordion} 
presents it in details. 







