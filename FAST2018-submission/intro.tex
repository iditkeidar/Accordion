
% NoSQL stores are popular
Persistent NoSQL key-value (KV-)stores such as Apache HBase~\cite{hbase} have become extremely popular over the last decade, 
and the range of applications for which they are used continuously increases. A small sample of  recently 
published use cases includes massive-scale online analytics (Airbnb/Airstream~\cite{airbnb}, 
Yahoo/Flurry~\cite{flurry}), product search 
and recommendation (Alibaba~\cite{alibabahbase}), 
graph storage (Facebook/Dragon~\cite{dragon}, 
Pinterest/Zen~\cite{zen}), and many more. 

% LSM trees optimize disk I/O 
The leading approach for implementating write-intensive key-value storage is \emph{log-structured merge (LSM)} trees~\cite{O'Neil:1996}.
This technology is ubiquitously used by popular key-value storage libraries (e.g., LevelDB~\cite{leveldb}, 
RocksDB~\cite{rocksdb}, ScyllaDB~\cite{scylladb}) and distributed storage systems on top 
of them (e.g., Bigtable~\cite{Chang2008}, Cassandra~\cite{cassandra}, HBase, 
MongoDB~\cite{mongodb}, MySql~\cite{mysql}). 
The premise for using LSM trees is that disk access is considered the principal bottleneck in storage systems, even with today's SSD hardware~\cite{rocksdb,Tanenbaum:2014:MOS:2655363,Wu:2012:AWB:2093139.2093140}. 
The main design goal is to improve write throughput, which LSM trees do by batching writes in memory 
and periodically \emph{flushing} the memory  to disk. This way, expensive random-access I/O is transformed 
into storage-friendly sequential I/O. 

% memory store and disk store
An LSM tree includes a \emph{memory store} in addition to the large \emph{disk store} consisting of a collection of files
(see Figure~\ref{fig:LSM}). 
The memory store absorbs writes, and is periodically flushed to disk as a new immutable file. Reads search the memory store
as well as the disk store. The number of files in the disk store has adverse impact on read performance. 
In order to reduce it, the system periodically runs a background \emph{compaction} process, which reads some files from 
disk and merges them into a single sorted one while removing redundant (overwritten or deleted) entries.
We provide further background on LSM trees and HBase in Section~\ref{sec:background}.

% Compactions hurt 
The performance of modern LSM stores is highly optimized, and yet as technology scales and real-time 
performance expectations increase, these systems face more stringent  demands. In particular, 
they are extremely sensitive to the rate and extent of compactions. If compactions are infrequent, read performance
suffers (because more files need to be searched and caching is less efficient when  data is scattered across multiple files). On the other hand, if 
compactions are too frequent, they jeopardize the performance of both writes and reads by consuming CPU and I/O resources. 
\remove{
In other words, LSM trees struggle to balance between {\em read amplification} (the number 
of disk reads per query) and {\em write amplification} (the ratio of bytes written to disk versus bytes written to the 
database)~\cite{rocksdbtuning}. 
}
In addition to their performance impact, frequent compactions increase the disk write volume, which  
accelerates device wear-out, especially for SSD hardware~\cite{Hu:2009}. 

% RAM is important too!
Not surprisingly, enormous efforts have been invested in compaction parameter tuning, scheduling, etc.~\cite{hbasetuning,
universalcompaction,scylladbcompaction,Sears:2012}. However, these approaches only deal with the aftermath
of organizing the disk store as a sequence of files created upon memory store overflow events, and do not 
address memory management. And yet, the amount of data written by memory flushes  
directly impacts the ensuing flush and compaction toll. 
While today's state-of-the-art LSM trees gravitate towards using large memory 
stores that make flushes less frequent (e.g., the HBase default memory store size is 128MB), 
they do not physically delete removed or overwritten data,
and therefore do not reduce the amount of data written to disk.
Moreover, state-of-the-art memory stores are organized as dynamic data structures consisting of small objects,
which becomes inefficient when the store size increases, in particular in managed languages like Java. 

% Drumroll 
We introduce \sys, an algorithm for memory store management in LSM trees. 
\sys\ re-applies the classic LSM tree design principles to the memory store. Namely, it partitions this store
 into a small dynamic segment that absorbs writes, and a sequence of static segments created 
 from previous dynamic segments by \emph{in-memory flushes}; the latter occur at a higher rate
 than disk flushes.  The static segments have an optimized memory
 layout (flat index), which reduces their footprint. The algorithm also 
 performs \emph{in-memory compactions}, which eliminate redundant %(removed or overwritten) 
 data before it is written to disk. As a result, \sys\ delays disk flushes, which both reduces 
 the write volume and improves read latency as more reads can be satisfied from RAM.
  \sys's data organization is illustrated in Figure~\ref{fig:accordion};
 Section~\ref{sec:accordion} fleshes out its details. 

% Status
\sys\ is implemented in HBase production code; it is generally available starting the HBase 2.0 release. 
An extensive evaluation process led to \sys's adoption as HBases's default memory store implementation. 

% Experiments
In Section~\ref{sec:eval} we experiment with the \sys\ HBase implementation in a range of scenarios.
We study production-size datasets and data layouts, with multiple storage types (SSD and HDD). 
We conclude that the algorithm's contribution to overall system performance is substantial, 
especially under a  heavy-tailed (Zipf) key access distribution with 
small objects, as occurs in many production use cases~\cite{Wu2015}. For example, \sys\/ 
improves the system's write throughput by up to $48\%$, and reduces read tail latency 
(in HDD settings) by up to $40\%$. At the same time, it reduces the disk flush and compaction write volume by up $30$\%. 
Surprisingly, we see that in most settings, disk I/O is \emph{not} the principal bottleneck. Rather, the memory management 
overhead is more substantial: the improvements are highly correlated with  reduction in \emph{garbage collection (GC)} time. 

In summary, \sys\ grows and shrinks the system's memory footprint like accordion bellows, giving it breathing 
room. Our experiments show that it significantly improves end-to-end system performance and reduces disk wear, 
which led to the recent adoption of this solution in HBase. Section~\ref{sec:related} discusses related work, 
and Section~\ref{sec:conclusions} concludes the paper.

