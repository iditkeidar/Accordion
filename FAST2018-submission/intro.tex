
% NoSQL stores are popular
NoSQL key-value data stores such as Apache HBase~\cite{hbase} have become extremely popular over the last decade, 
and the range of applications for which they are used continuously increases. A small sample of  recently 
published use cases includes massive-scale online analytics (Airbnb/Airstream\footnote{\small{\url{https://www.slideshare.net/HBaseCon/apache-hbase-at-airbnb}}}, 
Yahoo/Flurry\footnote{\small{\url{https://www.slideshare.net/HBaseCon/hbasecon-2015-hbase-operations-in-a-flurry}}}), e-commerce product search 
and recommendation (Alibaba~\cite{alibabahbase}), 
graph storage (Facebook/Dragon\footnote{\small{\url{https://code.facebook.com/posts/1737605303120405/dragon-a-distributed-graph-query-engine/}}}, 
Pinterest/Zen\footnote{\small{\url{https://www.slideshare.net/InfoQ/zen-pinterests-graph-storage-service}}}), and many more. 

\remove{ 
Similarly, Imgur uses HBase to power its notifications system\footnote{\url{https://dzone.com/articles/why-imgur-dropped-mysql-in-favor-of-hbase}};
Spotify uses HBase as base for Hadoop and machine learning jobs\footnote{\url{https://apachebigdata2015.sched.com}};
web mail metadata and search serving (Yahoo Mail), 
\footnote{\small{HBaseCon 2017 -- \url{http://hbase.apache.org/www.hbasecon.com}}}. 
}

% LSM trees optimize disk I/O 
The leading approach for implementation of write-intensive key-value storage is \emph{log-structured merge (LSM)} trees~\cite{O'Neil:1996}.
This technology is ubiquitously used by popular key-value storage libraries (e.g., LevelDB\footnote{\small{\url{https://github.com/google/leveldb}}}, 
RocksDB\footnote{\small{\url{https://rocksdb.org}}}, ScyllaDB\footnote{\small{\url{https://github.com/scylladb/scylla}}}) and distributed systems on top 
of them (e.g., Bigtable~\cite{Chang2008}, Cassandra\footnote{\small{\url{http://cassandra.apache.org/}}}, HBase, 
MongoDB\footnote{\small{\url{http://mongorocks.org/}}}, MySql\footnote{\small{\url{http://myrocks.io/}}}). 
The premise for using LSM trees is that disk access is considered the principal bottleneck in storage systems, even with today's SSD hardware~\cite{rocksdb,Tanenbaum:2014:MOS:2655363,Wu:2012:AWB:2093139.2093140}. 
The main design goal is to improve write throughput, which LSM trees do by batching writes in memory 
and periodically \emph{flushing} the memory  to disk. This way, expensive random-access I/O is transformed 
into storage-friendly sequential I/O. 

% memory store and disk store
An LSM tree includes a \emph{memory store} in addition to the large \emph{disk store} consisting of a collection of files. 
The memory store absorbs writes, and is periodically flushed to disk as a new immutable file. Reads search for  data
in the memory store, as well as in the disk store. The number of files has adverse impact on read performance. 
In order to reduce it, the system periodically runs a background \emph{compaction} process, which reads some files from 
the disk and merges them into a single sorted file while removing redundant (overwritten or deleted) entries.
We provide further background on LSM trees in Section~\ref{sec:background}.

% Compactions hurt 
The performance of modern LSM stores is highly optimized, and yet as technology scales and real-time 
performance expectations increase, these systems face more stringent performance demands. In particular, 
they are extremely sensitive to the rate and extent of compactions. If compactions are infrequent, read performance
suffers (e.g., caching is less efficient when the data is scattered across multiple files). On the other hand, if 
compactions are too frequent, they jeopardize the performance of both writes and reads by consuming CPU 
and I/O resources. In other words, LSM trees struggle to balance between {\em read amplification} (the number 
of disk reads per query) and {\em write amplification} (the ratio of bytes written to disk versus bytes written to the 
database). In addition to its performance impact, write amplification accelerates device wear-out, especially for SSD 
hardware~\cite{Hu:2009}. 

% RAM is important too!
Not surprisingly, enormous efforts have been invested in compaction parameter tuning, scheduling, etc.~\cite{hbasetuning,
universalcompaction,scylladbcompaction,Sears:2012}. However, these approaches only deal with the aftermath
of organizing the disk store as a sequence of files created upon memory store overflow events, and do not 
address memory management. And yet, the amount of data written by memory flushes  
directly impacts the ensuing flush and compaction toll. 
While today's state-of-the-art LSM trees gravitate towards using large memory 
stores that make flushes less frequent\footnote{For example, the HBase default memory store size is 128MB.}, 
they do not physically delete removed or overwritten data,
% or employ in-memory compaction, 
and therefore do not reduce the amount of data written to disk.
Moreover, state-of-the-art memory stores are organized as dynamic data structures consisting of small objects,
which becomes inefficient when the store size increases, in particular in managed languages like Java. 

% Drumroll 
We introduce \sys, an algorithm for memory store management in LSM trees. 
\sys\ re-applies the classic LSM tree design principles to the memory store. Namely, it partitions this store
 into a small dynamic segment that absorbs writes, and a sequence of static segments created 
 from previous dynamic segments by \emph{in-memory flushes}; memory flushes occur at a higher rate
 than their disk counterparts.  The static segments have an optimized memory
 layout (flat index), which reduces the memory overhead. Depending on the workload, the algorithm may 
 perform \emph{in-memory compactions}, which eliminate redundant (removed or over-written) data before 
 it is written to disk. 
 Section~\ref{sec:accordion}  fleshes out the implementation details. 
 
 \remove{
 In-memory flushes and compactions occur at a higher rate than their system-level siblings. They allow the system 
 flush to disk less frequently, and consequently reduce the overall I/O and the CPU cycles wasted on background maintenance
  of the LSM tree structure. In Java implementations (e.g., HBase), shrinking 
 the dynamic component results in significant reduction of GC overhead. Furthermore, since the static indices 
 are flat, the system can place them in off-JVM heap memory, thereby exploiting more RAM. Finally, reducing the 
 disk I/O decelerates the hardware wear-out period. 
% To Do: say something in conclusions about off heap
}

% Status
\sys\ is implemented in HBase production code; it is generally available starting the HBase 2.0 release. 
An extensive evaluation process led to \sys's adoption as default memory store implementation. 

% Experiments
In Section~\ref{sec:eval} we experiment with the \sys\ HBase implementation in a range of scenarios.
We study production-size datasets and data layouts, as well as multiple storage types (SSD and HDD). 
We conclude that the algorithm's contribution to the overall system performance is substantial, 
especially when working with small objects and heavy-tailed (Zipf) distributions. For example, \sys\/ 
improves the system write throughput by up to $40\%$, and reduces the tail read latency by up to 
\inred{XX\%}. At the same time, it reduces the write amplification by up to \inred{XX\%}. Surprisingly, we see 
that in most settings, disk I/O is \emph{not} the principal bottleneck. Rather, the memory management 
overhead is more substantial: the improvements are highly correlated with  reduction in garbage collection time. 

In summary, \sys\ grows and shrinks the system's memory footprint like accordion bellows, 
giving it breathing room. Our experiments show that this significantly improves end-to-end system performance, 
which led to the recent adoption of this solution in HBase. Section~\ref{sec:conclusions} concludes the paper.

