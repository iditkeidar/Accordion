\documentclass[letterpaper,twocolumn,10pt]{article}
%\usepackage{times}
\usepackage{usenix,epsfig,endnotes}

%========================
%  Packages
%========================

\usepackage{graphicx,url,color}
\usepackage{amsmath}
\usepackage{amssymb}
%\usepackage{amsthm}  %<---- for a different "look" in theorems (theorem word in bold, etc.) %ACM conflict with proof definition?
%\usepackage{subfigure}
%\usepackage[tight,footnotesize]{subfigure}
\usepackage{algorithm}
\usepackage[noend]{algpseudocode}
\usepackage{times}

%\usepackage{todonotes}
%\usepackage[normalem]{ulem} %strikethrough: \sout{Hello World}
%\usepackage{lastpage} %for number of pages
%\usepackage{xspace}
%\usepackage{multirow}
%\usepackage{balance}

\usepackage[colorlinks=true,allcolors=blue,breaklinks]{hyperref}   % hyperlinks, including DOIs and URLs in bibliography
\usepackage{subcaption}
\usepackage{graphicx}
%\usepackage{caption}
\usepackage{float}

%========================
%  Macros
%========================

\newcommand{\code}[1]{\textsf{\fontsize{9}{11}\selectfont #1}}

\newcommand{\inred}[1]{{\color{red}{#1}}}
\newcommand{\remove}[1]{}
\newcommand{\Idit}[1]{[[\inred{Idit: #1}]]}
\newcommand{\tb}{\hspace{5mm}}

\newcommand{\sys}{Accordion}


\newcommand{\speedup}[1]{#1$\times$}
\newcommand{\tuple}[1]{\ensuremath{\langle \mbox{#1} \rangle}}

%========================
\begin{document}

\date{}

\title{\Large \bf  Playing HBase Tunes on \sys}


\author{
\remove{
Edward Bortnikov\footnotemark[1]  \tb
Anastasia Braginsky\footnotemark[1]  \tb
Eshcar Hillel\footnotemark[1] \tb 
Idit Keidar\footnotemark[1] \footnotemark[2] \\
	\footnotemark[1] Yahoo Research\ \ \footnotemark[2] Technion  \\ [2mm]
%\small Submission Type: Research	
}
Deployed-system paper
} % end author


\maketitle



%=========================================================================
%  Abstract
%=========================================================================

\subsection*{Abstract}

Distributed NoSQL data stores are now ubiquitously used for a wide range of Internet-scale applications, 
including search indexing, analytics, user data management, messaging, advertizing, online trading, and much much more.
Such data stores are commonly organized as  log-structured merge (LSM) trees, 
which replace random disk writes with sequential I/O by accumulating large batches of updates 
in an in-memory data structure and merging it with  on-disk store in the background. 
The performance of state-of-the-art LSM stores is highly tuned, and yet as technology scales 
and real-time performance expectations from Internet applications increase, these systems face more stringent performance demands.

In this paper we report on our experience in  alleviating performance bottlenecks in Apache HBase. 
We present \sys, a new memory management scheme for LSM-based data stores, which we have implemented and committed in HBase 2.0.
\sys\ reduces the size of the highest LSM level, namely, the dynamic memory component that absorbs writes, and utilizes the remainging 
RAM capacity to store flat static data. This both saves space and reduces the memory management overhead (garbage collection, etc.). 
In addition, \sys\ allows for in-memory compactions (of the flat data), whereas existing LSM-based stores employ only disk compactions. 
\sys\ thus produces  fewer disk writes, which reduces disk leveling as well as the need for disk compactions. 
Perhaps surprisingly, our experiments show that memory management overhead is a more significant bottleneck than disk I/O in a
wide range of deployments and workloads, and hence \sys\ reaps more performance benefits from flattening than from in-memory compactions.



\remove{
such as  \Idit{other examples?} 

\emph{log-structured merge (LSM)} trees,

Example NoSQL data stores: Bigtable, HBase, LevelDB, MongoDB, SQLite4, RocksDB, WiredTiger,[5] Apache Cassandra, and InfluxDB.

1. Yahoo/Flurry - mobile app analytics. Yahoo - Mail metadata and search 
2. Pinterest - user data and graph service
3. Alibaba - chat and online trade data
4. Facebook - messaging


}
%========================

\section{Introduction} \label{sec:intro}

% NoSQL stores are popular
NoSQL key-value data stores such as Apache HBase~\cite{hbase} have become extremely popular over the last decade, 
and the range of applications for which they are used continously increases. A small sample of the recently 
published use cases includes mass-scale online analytics (Airbnb/Airstream\footnote{\small{\url{https://www.slideshare.net/HBaseCon/apache-hbase-at-airbnb}}}, 
Yahoo/Flurry\footnote{\small{\url{https://www.slideshare.net/HBaseCon/hbasecon-2015-hbase-operations-in-a-flurry}}}), e-commerce product search 
and recommendation (Alibaba\footnote{\small{\url{https://blog.cloudera.com/blog/2017/03/offheap-read-path-in-production-the-alibaba-story/}}}), 
graph storage (Facebook/Dragon\footnote{\small{\url{https://code.facebook.com/posts/1737605303120405/dragon-a-distributed-graph-query-engine/}}}, 
Pinterest/Zen\footnote{\small{\url{https://www.slideshare.net/InfoQ/zen-pinterests-graph-storage-service}}}), etc. 

%Similarly, Imgur uses HBase to power its notifications system\footnote{\url{https://dzone.com/articles/why-imgur-dropped-mysql-in-favor-of-hbase}};
%Spotify uses HBase as base for Hadoop and machine learning
%jobs\footnote{\url{https://apachebigdata2015.sched.com}};
%web mail metadata and search serving (Yahoo Mail), 
%\footnote{\small{HBaseCon 2017 -- \url{http://hbase.apache.org/www.hbasecon.com}}}. 

% LSM trees optimize disk i/o 
The leading approach for implementation of scalable key-value storage is \emph{log-structured merge (LSM)} trees~\cite{O'Neil:1996}.
This technology is ubiquitously used by popular key-value storage libraries (e.g., LevelDB, RocksDB, ScyllaDB) and distributed systems on top 
of them (e.g., Bigtable~\cite{Chang2008}, Cassandra\footnote{\small{\url{http://cassandra.apache.org/}}}, HBase, 
MongoDB\footnote{\small{\url{http://mongorocks.org/}}}, MySql\footnote{\small{\url{http://myrocks.io/}}}), etc. 
%which are employed by  Apache HBase, Bigtable, RocksDB, LevelDB, MongoDB, SQLite4, WiredTiger, Apache Cassandra,  InfluxDB, and many more.
The premise for using LSM trees is that disk access is considered the principal bottleneck in storage systems, even with today's SSD hardware~\cite{rocksdb,Tanenbaum:2014:MOS:2655363,Wu:2012:AWB:2093139.2093140}. 
The main design goal is to improve write throughput, which LSM trees do by batching writes in memory 
and periodically \emph{flushing} the memory  to disk. This way, random-access I/O is transformed to storage-friendly sequential I/O. 

An LSM tree includes a \emph{memory store} in addition to the large \emph{disk store} consisting of a collection of files. 
The memory store absorbs writes, and is periodically flushed to disk as a new ordered file. Reads search for the data
in the memory store, as well as in the disk store. The number of files has adverse impact on read performance. 
In order to reduce it, the system peridically runs a background \emph{compaction} process, which reads some files from 
the disk and merges them into a single file while removing redundant (overwritten or deleted) entries.%. For further background, see Section~\ref{sec:background}.

% Compactions hurt 
The performance of modern LSM stores is highly optimized, and yet as technology scales and real-time 
performance expectations increase, these systems face more stringent performance demands. In particular, 
they are extremely sensitive to the rate and extent of compactions. If compactions are infrequent, read performance
suffers (e.g., caching is less efficient when the data is scattered across multiple files). On the other hand, if 
compactions are too frequent, they jeopardize the performance of both writes and reads by consuming CPU 
and I/O resources. In formal terms, LSM trees struggle to balance between {\em read amplification} (the number 
of disk reads per query) and {\em write amplification} (the ratio of bytes written to disk versus bytes written to the 
database). In addition to its performance impact, write amplification accelerates device wearout, especially for SSD 
hardware~\cite{Hu:2009}. 


 
%One of the pain points with LSM stores is compactions \Idit{Can we cite something from RocksDB or another KV store here?}.
In HBase, a major pain point is the occurrance of so-called \emph{major compactions}, which replace all the files of a given table.
Indeed, HBase manuals and tuning tips are often aimed at reducing or even avoiding major compactions altogether.
 %\footnote{\url{https://blogs.msdn.microsoft.com/ashish/2016/09/02}}.

% Memory management
LSM trees can benefit from a large memory store, which absorbs many writes before each disk flush, 
and in turn delays the need for disk compactions, \Idit{Citations for this?} including major ones. 
The memory store is \emph{dynamic}, i.e., constantly changes, and must support fast lookup by key. 
It is therefore typically organized as a skip list
\Idit{Citations for this?} or a similar data structure.  Unfortunately, managing a large dynamic data structure induces 
non-negligble  overhead. 
Moreover, 
LSM-based data stores typically perform compactions only on disk and do not attempt to eliminate redundancies in the memory store.
Given that data access patterns often follow heavy-tailed distributions like Zipf,  this means that many versions of popular objects are 
wastefully kept in the in-memory data structure. 
Our work is motivated by this management overhead and waste. 

% Drumroll 
In Section~\ref{sec:accordion}
we introduce \sys, a new memory management scheme for LSM trees.
The principles behind \sys\ are 
\begin{enumerate}
\item \emph{in-memory compactions} can free up memory space to absorb more writes before the
memory store needs to be flushed, thus reducing compactions and disk wear; and
\item \emph{flattening} dynamic data structures into static buffers can reduce memory management overhead.
\end{enumerate}

% Implementation
In Section~\ref{sec:hbase} we  discuss on our implementation of  \sys\ in HBase, which is committed  as the default path in HBase 2.0. 
\Idit{Say something interesting about this process?}
% Evaluation
In Section~\ref{sec:eval} we experiment with \sys\ in a range of  scenarios -- with SSD and HDD storage, with different numbers
of regions, and different workloads. Surprisingly, we see that in virtually all settings, disk I/O is \emph{not} the principal performance bottleneck,
but rather memory management overhead is more substantial. By flattening the memory store \sys\ reduces this overhead and improves overall performance by \inred{XX\% to XX\%}. In-memory compactions have minor impact on performance at best, but do reduce 
write amplification by \inred{XX\% to XX\%}, which is important for extending disk life.

\Idit{Need to think of a good summary sentence for here. }




 


\section{Conclusion} \label{sec:conclusions}


%\subsection*{Acknowledgments}

\newpage

\bibliographystyle{acm}

\bibliography{refs}

\end{document}
