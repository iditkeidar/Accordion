\documentclass[letterpaper,twocolumn,10pt]{article}
%\usepackage{times}
\usepackage{usenix,epsfig,endnotes}

%========================
%  Packages
%========================

\usepackage{graphicx,url,color}
\usepackage{amsmath}
\usepackage{amssymb}
%\usepackage{amsthm}  %<---- for a different "look" in theorems (theorem word in bold, etc.) %ACM conflict with proof definition?
%\usepackage{subfigure}
%\usepackage[tight,footnotesize]{subfigure}
\usepackage{algorithm}
\usepackage[noend]{algpseudocode}
\usepackage{times}

%\usepackage{todonotes}
%\usepackage[normalem]{ulem} %strikethrough: \sout{Hello World}
%\usepackage{lastpage} %for number of pages
%\usepackage{xspace}
%\usepackage{multirow}
%\usepackage{balance}

\usepackage[colorlinks=true,allcolors=blue,breaklinks]{hyperref}   % hyperlinks, including DOIs and URLs in bibliography
\usepackage{subcaption}
%\usepackage{caption}
\usepackage{float}

%========================
%  Macros
%========================

\newcommand{\code}[1]{\textsf{\fontsize{9}{11}\selectfont #1}}

\newcommand{\inred}[1]{{\color{red}{#1}}}
\newcommand{\remove}[1]{}
\newcommand{\Idit}[1]{[[\inred{Idit: #1}]]}
\newcommand{\eshcar}[1]{[[\inred{eshcar: #1}]]}
\newcommand{\tb}{\hspace{5mm}}

\newcommand{\sys}{Accordion}
\newcommand{\basic}{{\em Basic}}
\newcommand{\eager}{{\em Eager}}
\newcommand{\adp}{{\em Adaptive}}
\newcommand{\none}{{\em None}}
\newcommand{\magic}{{\em Adaptive}}
\newcommand{\figw}{0.925\columnwidth}


\newcommand{\speedup}[1]{#1$\times$}
\newcommand{\tuple}[1]{\ensuremath{\langle \mbox{#1} \rangle}}

%========================
\begin{document}

\date{}

\title{\Large \bf  Playing HBase Tunes on \sys}


\author{
\remove{
Edward Bortnikov\footnotemark[1]  \tb
Anastasia Braginsky\footnotemark[1]  \tb
Eshcar Hillel\footnotemark[1] \tb 
Idit Keidar\footnotemark[1] \footnotemark[2] \\
	\footnotemark[1] Yahoo Research\ \ \footnotemark[2] Technion  \\ [2mm]
%\small Submission Type: Research	
}
Deployed-system paper
} % end author

\maketitle



%=========================================================================
%  Abstract
%=========================================================================

\subsection*{Abstract}

Log-sequenced-merge (LSM) trees have emerged as the technology of choice for building scalable 
write-intensive key-value storage systems. An LSM tree replaces random I/O with sequential 
I/O by accumulating large batches of writes in a \emph{memory store} prior to flushing them to log-structured 
disk storage; the latter is continuously re-organized in the background through a \emph{compaction}
process for efficiency of reads. Though inherent to the LSM design, frequent compactions are a major pain point 
because they slow down data store operations, primarily writes, and also increase disk wear.

We argue that this pain point may be mitigated via better  organization of the memory store.
We present \sys\ -- an algorithm that addresses this problem by re-applying
the battle-tested LSM tree design principles to memory management. 
\sys\ is implemented in the production code of Apache HBase, where it became the default deployment option 
after extensive evaluation. We demonstrate \sys's double-digit performance gains versus 
the baseline HBase implementation, and discuss some unexpected lessons learned. 

%========================

\section{Introduction} \label{sec:intro}

% NoSQL stores are popular
NoSQL key-value data stores such as Apache HBase~\cite{hbase} have become extremely popular over the last decade, 
and the range of applications for which they are used continously increases. A small sample of the recently 
published use cases includes mass-scale online analytics (Airbnb/Airstream\footnote{\small{\url{https://www.slideshare.net/HBaseCon/apache-hbase-at-airbnb}}}, 
Yahoo/Flurry\footnote{\small{\url{https://www.slideshare.net/HBaseCon/hbasecon-2015-hbase-operations-in-a-flurry}}}), e-commerce product search 
and recommendation (Alibaba\footnote{\small{\url{https://blog.cloudera.com/blog/2017/03/offheap-read-path-in-production-the-alibaba-story/}}}), 
graph storage (Facebook/Dragon\footnote{\small{\url{https://code.facebook.com/posts/1737605303120405/dragon-a-distributed-graph-query-engine/}}}, 
Pinterest/Zen\footnote{\small{\url{https://www.slideshare.net/InfoQ/zen-pinterests-graph-storage-service}}}), etc. 

%Similarly, Imgur uses HBase to power its notifications system\footnote{\url{https://dzone.com/articles/why-imgur-dropped-mysql-in-favor-of-hbase}};
%Spotify uses HBase as base for Hadoop and machine learning
%jobs\footnote{\url{https://apachebigdata2015.sched.com}};
%web mail metadata and search serving (Yahoo Mail), 
%\footnote{\small{HBaseCon 2017 -- \url{http://hbase.apache.org/www.hbasecon.com}}}. 

% LSM trees optimize disk i/o 
The leading approach for implementation of scalable key-value storage is \emph{log-structured merge (LSM)} trees~\cite{O'Neil:1996}.
This technology is ubiquitously used by popular key-value storage libraries (e.g., LevelDB, RocksDB, ScyllaDB) and distributed systems on top 
of them (e.g., Bigtable~\cite{Chang2008}, Cassandra\footnote{\small{\url{http://cassandra.apache.org/}}}, HBase, 
MongoDB\footnote{\small{\url{http://mongorocks.org/}}}, MySql\footnote{\small{\url{http://myrocks.io/}}}), etc. 
%which are employed by  Apache HBase, Bigtable, RocksDB, LevelDB, MongoDB, SQLite4, WiredTiger, Apache Cassandra,  InfluxDB, and many more.
The premise for using LSM trees is that disk access is considered the principal bottleneck in storage systems, even with today's SSD hardware~\cite{rocksdb,Tanenbaum:2014:MOS:2655363,Wu:2012:AWB:2093139.2093140}. 
The main design goal is to improve write throughput, which LSM trees do by batching writes in memory 
and periodically \emph{flushing} the memory  to disk. This way, random-access I/O is transformed to storage-friendly sequential I/O. 

An LSM tree includes a \emph{memory store} in addition to the large \emph{disk store} consisting of a collection of files. 
The memory store absorbs writes, and is periodically flushed to disk as a new ordered file. Reads search for the data
in the memory store, as well as in the disk store. The number of files has adverse impact on read performance. 
In order to reduce it, the system peridically runs a background \emph{compaction} process, which reads some files from 
the disk and merges them into a single file while removing redundant (overwritten or deleted) entries.%. For further background, see Section~\ref{sec:background}.

% Compactions hurt 
The performance of modern LSM stores is highly optimized, and yet as technology scales and real-time 
performance expectations increase, these systems face more stringent performance demands. In particular, 
they are extremely sensitive to the rate and extent of compactions. If compactions are infrequent, read performance
suffers (e.g., caching is less efficient when the data is scattered across multiple files). On the other hand, if 
compactions are too frequent, they jeopardize the performance of both writes and reads by consuming CPU 
and I/O resources. In formal terms, LSM trees struggle to balance between {\em read amplification} (the number 
of disk reads per query) and {\em write amplification} (the ratio of bytes written to disk versus bytes written to the 
database). In addition to its performance impact, write amplification accelerates device wearout, especially for SSD 
hardware~\cite{Hu:2009}. 


 
%One of the pain points with LSM stores is compactions \Idit{Can we cite something from RocksDB or another KV store here?}.
In HBase, a major pain point is the occurrance of so-called \emph{major compactions}, which replace all the files of a given table.
Indeed, HBase manuals and tuning tips are often aimed at reducing or even avoiding major compactions altogether.
 %\footnote{\url{https://blogs.msdn.microsoft.com/ashish/2016/09/02}}.

% Memory management
LSM trees can benefit from a large memory store, which absorbs many writes before each disk flush, 
and in turn delays the need for disk compactions, \Idit{Citations for this?} including major ones. 
The memory store is \emph{dynamic}, i.e., constantly changes, and must support fast lookup by key. 
It is therefore typically organized as a skip list
\Idit{Citations for this?} or a similar data structure.  Unfortunately, managing a large dynamic data structure induces 
non-negligble  overhead. 
Moreover, 
LSM-based data stores typically perform compactions only on disk and do not attempt to eliminate redundancies in the memory store.
Given that data access patterns often follow heavy-tailed distributions like Zipf,  this means that many versions of popular objects are 
wastefully kept in the in-memory data structure. 
Our work is motivated by this management overhead and waste. 

% Drumroll 
In Section~\ref{sec:accordion}
we introduce \sys, a new memory management scheme for LSM trees.
The principles behind \sys\ are 
\begin{enumerate}
\item \emph{in-memory compactions} can free up memory space to absorb more writes before the
memory store needs to be flushed, thus reducing compactions and disk wear; and
\item \emph{flattening} dynamic data structures into static buffers can reduce memory management overhead.
\end{enumerate}

% Implementation
In Section~\ref{sec:hbase} we  discuss on our implementation of  \sys\ in HBase, which is committed  as the default path in HBase 2.0. 
\Idit{Say something interesting about this process?}
% Evaluation
In Section~\ref{sec:eval} we experiment with \sys\ in a range of  scenarios -- with SSD and HDD storage, with different numbers
of regions, and different workloads. Surprisingly, we see that in virtually all settings, disk I/O is \emph{not} the principal performance bottleneck,
but rather memory management overhead is more substantial. By flattening the memory store \sys\ reduces this overhead and improves overall performance by \inred{XX\% to XX\%}. In-memory compactions have minor impact on performance at best, but do reduce 
write amplification by \inred{XX\% to XX\%}, which is important for extending disk life.

\Idit{Need to think of a good summary sentence for here. }




 


\section{Background} \label{sec:background}

% Functionality
HBase is a distributed key-value store that is part of the Hadoop open source  technology suite. 
It is implemented in Java. Similarly to other modern KV-stores, HBase follows the design of 
Google's Bigtable~\cite{Chang2008}. We sketch out their common design principles 
and terminology. 

\paragraph{Data model}
KV-stores hold data items (referred to as \emph{rows}) identified by unique 
row keys. Each row can consist of multiple fields (\emph{columns}) identified by unique 
column keys. Co-accessed columns (typically used by the same application) can be 
aggregated into  \emph{column families} to optimize access. The data is multi-versioned, 
i.e., multiple versions of the same row key can exist, each identified by a unique {\em timestamp}. 
The smallest unit of data, named {\em cell}, is defined by a combination of a row key, a
column key, and a timestamp.

The basic KV-store API includes \emph{put} (point update of one or more cells, by row key), 
\emph{get} (point query of one or more columns, by row key, and possibly column keys), 
and \emph{scan} (range query of one or more columns, of all keys between 
an ordered pair of begin and end keys). 

\begin{figure}[tb]
\center
\includegraphics[width=0.9\columnwidth]{LSM} 
\caption{A log-sequence-merge LSM tree consists of a small memory store ({\em MemStore} in HBase) 
and a large disk store (collection of {\em HFile}s). Put operations update the MemStore. The latter is 
double-buffered: a flush creates an immutable snapshot of the active buffer, and creates a new active 
one. The snapshot is then written to disk in the background.}
\label{fig:LSM}
\end{figure}

\paragraph{Data management}
KV-stores achieve scalability by sharding tables into range partitions by row key. 
A shard is called {\em region\/} in HBase (\emph{tablet} in Bigtable). 
A region is a collection of \emph{stores}, each associated with a column family. 
Each store is backed by a collection of files sorted by row key, called \emph{HFile}s in HBase 
(\emph{sst files} in Bigtable). 

In production, HBase is typically layered on top of Hadoop Distributed Filesystem (HDFS), 
which provides a reliable file storage abstraction. HDFS and HBase normally share the same hardware. 
Both scale horizontally to thousands of nodes. HDFS replicates data for availability (3-way by default). 
It is optimized for very large files (the default block size is $64$MB).

Data access in HBase is provided through {\em region servers} (analogous to {\em tablet servers}
in Bigtable). Each region server controls multiple stores (tens to hundreds in production settings). 
For locality of access, the HBase management plane (\emph{master server}) tries to lay out the 
HFiles in HDFS so that their primary replicas are colocated with the region servers that control them. 

\paragraph{LSM trees}
Each store is organized as an LSM tree, which collects writes in a memory store 
(\emph{MemStore} in HBase), and periodically flushes the memory into a disk store, as illustrated in 
Figure~\ref{fig:LSM}. Each flush creates a new immutable HFile, ordered by row key for query efficiency. 
HFiles are created big, hence flushes are relatively infrequent. 

To allow puts to proceed in parallel with I/O, MemStore employs double buffering:
it maintains a dynamic \emph{active} buffer absorbing puts, and a static snapshot
that holds the previous version of the active buffer. The latter is written to the 
filesystem in the background. Both buffers are ordered by key, as are the HFile.  

% LSM puts and flushes
A flush occurs  when either the active buffer 
exceeds the region size limit ($128$MB by default), or when the overall footprint of all MemStore's
in the region server exceeds the global size limit ($40\%$ of the heap by default). 
\ingreen{Flush first shifts the current active buffer, to be the snapshot (making it immutable) and creates a new empty active buffer.}
%Flush first freezes the current active buffer, making it immutable, and creates a new empty one.
It then replaces the reference to the active buffer, and proceeds to write the \ingreen{snapshot}
as a new HFile. 

% WAL
In order to guarantee durability of writes between flushes, updates are first written to 
a \emph{write-ahead log} (WAL). HBase implements the logical WAL as collection of one or more physical 
logs per region server, called \emph{HLog}s, each consisting of multiple log files on HDFS. 
As new data gets flushed to HFiles, the old log files become redundant, and the system collects 
them in the background. On the other hand, if some HLog grows too big, the system may forcefully
flush before the memory bound is reached, in order to enable the log's truncation. 
Since logged data is only required at recovery time and is only scanned sequentially, HLogs
may be stored on slower devices than HFiles in production (e.g., HDDs vs SSDs). 
Real-time applications often trade durability for speed by aggregating multiple log records 
in memory prior to asynchronously writing them to WAL in a bulk. 

To keep the number of HFiles per store bounded, \emph{compactions} merge multiple files 
into one, while eliminating redundant (overwritten) versions. If compactions cannot keep up
with the flush rate, HBase may either throttle or totally block the puts until the compactions 
successfully reduce the number of files. 

Most compactions are \emph{minor}, 
in the sense that they merge part of the \ingreen{HFiles}. \emph{Major} compactions merge all the region's 
\ingreen{HFiles}. Because they have a global view of the data, major compactions also eliminate 
{\em tombstone} versions that indicate row deletions. Typically, major compactions incur huge 
performance impact on concurrent operations. In production, they are either carefully scheduled 
or performed manually~\cite{hbasetuning}.

% LSM gets
The get operation searches for the key in parallel in the MemStore (both the active and the 
snapshot buffers) and in the HFiles. The search in HFiles is expedited through Bloom 
filters~\cite{Chang2008}, which eliminate most redundant disk reads. The system 
uses a large RAM cache for popular HFile blocks (by default $40\%$ of the heap).

\paragraph{Memory organization}
The MemStore's active buffer is traditionally implemented as a dynamic index over a collection of cells.  
The HBase implementation uses the standard Java concurrent skiplist map~\cite{javaskiplist}.
Data is multi-versioned -- every put creates a new immutable version of the row it is applied to, 
consisting of one or more cells. 

This implementation suffers from two drawbacks. First, the use of a big dynamic data structure entails 
the abundance of auxiliary small objects and references, which inflate the in-memory index and induce 
a high GC cost. The overhead is most significant when the managed objects are small, i.e., 
the metadata-to-data ratio is big~\cite{Wu2015}. Second, the versioning mechanism 
makes no attempt to eliminate redundancies prior to flush, i.e., the MemStore size  steadily grows, 
independently of the workload. This is especially wasteful for heavy-tailed (e.g., power law) distributions 
that are prevalent in production workloads~\cite{Devineni:2015}. 

The \sys\/ algorithm takes care of precisely these problems, to boost the 
LSM tree performance and reduce the storage wearout. Section~\ref{sec:accordion} 
presents it in detail. 







 

\section{\sys} \label{sec:accordion}
\subsection{Overview}

\begin{figure*}
\caption{\bf{LSM architecture based on compacting memstore.}}
\label{fig:compacting}
\end{figure*}

\sys\/ introduces a {\em compacting\/} memstore to the LSM tree design framework. Contrast to the traditional memstore, 
which maintains all the RAM-resident data in a single monolithic data structure, \sys\/ manages the data as a sequence of 
{\em segments}, ordered by creation time. At all times, the last created segment, called {\em active}, is mutable; it absorbs 
the put operations. The rest of the segments are immutable. The get and scan operations retrieve data from all the segments 
simultaneously, similarly to traditional LSM tree read from multiple levels. Figure~\ref{fig:compacting} illustrates the architecture. 

Once the active segment grows to a $\rho$ fraction of the memstore size bound, an {\em in-memory flush} happens.
The segment becomes immutable. The system queues it up to the list of inactive segments, called {\em pipeline}, 
and creates a new active segment. The  memstore periodically shrinks the pipeline in the background, by applying 
one or more {\em in-memory compaction} mechanisms: 
\begin{enumerate}
\item {\em Flattening of intra-segment search indices.} Since pipelined segments are immutable, it is sufficient to maintain 
their indices as compact ordered arrays, rather than reference-hungry skiplists. An additional advantage of the flat layout 
is that the index can be relocated to off-heap memory (unmanaged by the JVM), which improves performance predictability 
through reduced GC jitter~\cite{alibabahbase}. 
\item {\em Merging multiple segment indices.} A single index covering multiple segment data is created. Redundant data 
versions are not eliminated, to avoid comparisons and physical data copy. 
\item  {\em Merging multiple segment data.} Extends the above with redundant data elimination. The created flat index 
contains no redundancies. In case the memstore manages its cell data storage internally, the surviving cells are relocated
to new slabs; otherwise, the redundant cells are simply de-referenced, and later garbage-collected.    
\end{enumerate} 
The choice of compaction mechanisms is guided by the policies described in Section~\ref{sec:policies}. 

When the region server flushes a compacting memstore to disk, it scans the data from a snapshot 
of the pipelined segments. Similarly to the traditional design, the scan eliminates the redundancies 
prior to flush. 
 
%Summing up, in contrast with traditional memstores that can only grow between disk flushes, compacting memstores 
%can both expand and contract, resembling the movement of accordion bellows. 
%In-memory compactions that reduce the number of immutable segments bound the tail read latencies, as most reads access 
%a few segments. 

\subsection{Compaction Policies}
\label{sec:policies}

We implement three in-memory compaction policies: 
\paragraph{\basic} (low-overhead). Once a segment becomes immutable, flatten its index. Once the pipeline size exceeds $s$, 
merge all segment indices into one.  
\paragraph{\eager} (high-overhead, high-reward under redundancy-heavy workloads). 
In addition to \basic\/ mechanisms, eliminates data redundancies.
\paragraph{\adp} (the best of all worlds). The heuristic adapts to the recent workload, and selects either \basic\/ 
or \eager\/ optimizations depending on the context. \inred{Eshcar, please add the details}. 

\subsection{Implementation Details}

A compacting memstore is comprised from the active segment and a double-ended queue (pipeline) of inactive segments. 
The pipeline is accessed by read API's (get and scan), as well as by backgound disk flushes and in-memory compactions. 
The latter two modify the pipeline, by adding/removing/replacing segments. These modifications happen infrequently. 

The pipeline readers and writers coordinate through a lightweight copy-on-write, as follows. The pipeline object is versioned. 
Each modification promotes the version number, and atomically swaps the global reference to the new version clone. Note
that cloning is inexpensive -- only the segment references are copied since the segments themselves are immutable. 

The reads access the segments lock-free, through the version obtained at the beginning of the operation. If a disk flush
is scheduled in the middle of a read, a segment may migrate from the pipeline to the pre-flush snapshot buffer. The read 
safety is guaranteed by scanning the pipeline clone first. This way, a segment may be encountered twice but no data is 
lost. The scan algorithm filters out the duplicates. 

In-memory compaction is a read-modify-write operation, which swaps one or more segments in the pipeline 
with a new segment built from their data. This operation's atomicity is guaranteed by compare-and-swap (CAS), 
which flips pipeline version only if the latter did not change since the compaction started. Note that it is acceptable
for in-memory compaction to fail sometimes because it is an optimization that may be retried later. 

 

\section{Performance Study} \label{sec:eval}
We compare the performance of 3 memory management strategies 
\begin{description}
\item[\none] -- which does no compaction of the data, this was previously the default in HBase.
\item[\basic] -- which reduces the representation of the index, this is the default in HBase 2.0.
\item[\magic] -- which compact both the data and index and is adaptive to the workload behavior, this is still an experimental feature.
\end{description}

We compare the strategies by measuring their throughput and latency. In addition we measure the write volume, that is the number of KB written to the file system, and the cumulative gc time of each run.
We also compare different setting of the same strategy in order to find the optimal configuration.

\paragraph{Experiment setup.}

Our experiments run on 2  clusters with different hardware. 
The first cluster consists of five 12-core Intel Xeon 5 machines with 46GB RAM and 4TB 
SSD storage, interconnected by 1G Ethernet. 
The second cluster consists of five 8-core 
Intel Xeon E5620 servers with 24GB RAM and 1TB magnetic drive. The interconnects  are 1Gbps Ethernet. 
We denote these clusters as the SSD cluster and HDD cluster, respectively.
In both clusters we allocate three nodes to HBase nodes, 1 master and 2 region servers, one to simulate the client whose performance we measure, and one to simulate background traffic
as explained below. Each HBase node runs both an HBase region server or master within 8GB JVM containers and the underlying 
Hadoop File System (HDFS) server . 

The traffic is driven by a client running the popular YCSB benchmark~\cite{Cooper}. 
We run write-only and mixed read-write workload where keys are chosen either from either a zipfian or uniform distribution over 100 millions keys.
In all the experiments we create a table with 4 columns (in a single column family). Each update operation writes 25 Bytes values to all 4 columns of a single key, namely writing 100 Bytes of data.


\paragraph{GC overhead.}

Our experiment show that in write-only workloads gc and throughput have a very high correlation. 
Figure~\ref{fig:gc-throughput-log2} plots the scatter graph of cumulative gc time vs write throughput of several write-only experiments of the system with varying settings in different strategies. It clearly shows that  the lower the gc overhead is, the higher the throughput is.
Specifically, there is a negative linear correlation between their values on a log-log scale.

\begin{figure}[htb]
\includegraphics[width=\figw]{Figs/gc-throughput-log2.png}
\caption{{\bf GC-Throughput correlation.} Log-log scale plot of the cumulative gc time vs write throughput in different settings shows negative linear correlation.
}
\label{fig:gc-throughput-log2}
\end{figure}

\section{Conclusions} \label{sec:conclusions}


\subsection*{Acknowledgments}
Omitted for blind review.

\newpage

\bibliographystyle{acm}

\bibliography{refs}

\end{document}
