\documentclass[letterpaper,twocolumn,10pt]{article}
%\usepackage{times}
\usepackage{usenix,epsfig,endnotes}

%========================
%  Packages
%========================

\usepackage{graphicx,url,color}
\usepackage{amsmath}
\usepackage{amssymb}
%\usepackage{amsthm}  %<---- for a different "look" in theorems (theorem word in bold, etc.) %ACM conflict with proof definition?
%\usepackage{subfigure}
%\usepackage[tight,footnotesize]{subfigure}
\usepackage{algorithm}
\usepackage[noend]{algpseudocode}
\usepackage{times}

%\usepackage{todonotes}
%\usepackage[normalem]{ulem} %strikethrough: \sout{Hello World}
%\usepackage{lastpage} %for number of pages
%\usepackage{xspace}
%\usepackage{multirow}
%\usepackage{balance}

\usepackage[colorlinks=true,allcolors=blue,breaklinks]{hyperref}   % hyperlinks, including DOIs and URLs in bibliography
\usepackage{subcaption}
%\usepackage{caption}
\usepackage{float}

%========================
%  Macros
%========================

\newcommand{\code}[1]{\textsf{\fontsize{9}{11}\selectfont #1}}

\newcommand{\inred}[1]{{\color{red}{#1}}}
\newcommand{\remove}[1]{}
\newcommand{\Idit}[1]{[[\inred{Idit: #1}]]}
\newcommand{\tb}{\hspace{5mm}}

\newcommand{\sys}{Accordion}
\newcommand{\basic}{{\em Basic}}
\newcommand{\eager}{{\em Eager}}
\newcommand{\adp}{{\em Adaptive}}

\newcommand{\speedup}[1]{#1$\times$}
\newcommand{\tuple}[1]{\ensuremath{\langle \mbox{#1} \rangle}}

%========================
\begin{document}

\date{}

\title{\Large \bf  Playing HBase Tunes on \sys}


\author{
\remove{
Edward Bortnikov\footnotemark[1]  \tb
Anastasia Braginsky\footnotemark[1]  \tb
Eshcar Hillel\footnotemark[1] \tb 
Idit Keidar\footnotemark[1] \footnotemark[2] \\
	\footnotemark[1] Yahoo Research\ \ \footnotemark[2] Technion  \\ [2mm]
%\small Submission Type: Research	
}
Deployed-system paper
} % end author


\maketitle



%=========================================================================
%  Abstract
%=========================================================================

\subsection*{Abstract}

Distributed NoSQL data stores are now ubiquitously used for a wide range of Internet-scale applications, 
including search indexing, analytics, user data management, messaging, advertizing, online trading, and much much more.
Such data stores are commonly organized as  log-structured merge (LSM) trees, 
which replace random disk writes with sequential I/O by accumulating large batches of updates 
in an in-memory data structure. 
\remove{
and merging it with  on-disk store in the background. 
}

This paper  reports on our experience in improving the memory organization of the popular 
Apache HBase LSM-based NoSQL store. 
We present \sys, a new memory management scheme for LSM trees, which we have implemented and 
committed in HBase 2.0.
\sys\ reduces the size of the data structure holding the 
 top LSM level, namely, the dynamic memory component that absorbs writes, and utilizes the remainging 
RAM capacity to store flat static data. This  saves space and also reduces  memory management overhead (garbage collection, etc.). 
In addition, \sys\ allows for in-memory compactions (of the flat data), whereas existing LSM-based stores compact only on-disk data. 
\sys\ thus produces  fewer disk writes, which reduces disk wear as well as disk compactions. 

Perhaps surprisingly, our experiments show that memory management overhead is a more significant bottleneck than disk I/O in a
wide range of deployments and workloads, and hence \sys\ reaps more performance benefits from flattening than from in-memory compactions,
while the latter is important for extending disk lifes.

%========================

\section{Introduction} \label{sec:intro}

% NoSQL stores are popular
NoSQL key-value data stores such as Apache HBase~\cite{hbase} have become extremely popular over the last decade, 
and the range of applications for which they are used continously increases. A small sample of the recently 
published use cases includes mass-scale online analytics (Airbnb/Airstream\footnote{\small{\url{https://www.slideshare.net/HBaseCon/apache-hbase-at-airbnb}}}, 
Yahoo/Flurry\footnote{\small{\url{https://www.slideshare.net/HBaseCon/hbasecon-2015-hbase-operations-in-a-flurry}}}), e-commerce product search 
and recommendation (Alibaba\footnote{\small{\url{https://blog.cloudera.com/blog/2017/03/offheap-read-path-in-production-the-alibaba-story/}}}), 
graph storage (Facebook/Dragon\footnote{\small{\url{https://code.facebook.com/posts/1737605303120405/dragon-a-distributed-graph-query-engine/}}}, 
Pinterest/Zen\footnote{\small{\url{https://www.slideshare.net/InfoQ/zen-pinterests-graph-storage-service}}}), etc. 

%Similarly, Imgur uses HBase to power its notifications system\footnote{\url{https://dzone.com/articles/why-imgur-dropped-mysql-in-favor-of-hbase}};
%Spotify uses HBase as base for Hadoop and machine learning
%jobs\footnote{\url{https://apachebigdata2015.sched.com}};
%web mail metadata and search serving (Yahoo Mail), 
%\footnote{\small{HBaseCon 2017 -- \url{http://hbase.apache.org/www.hbasecon.com}}}. 

% LSM trees optimize disk i/o 
The leading approach for implementation of scalable key-value storage is \emph{log-structured merge (LSM)} trees~\cite{O'Neil:1996}.
This technology is ubiquitously used by popular key-value storage libraries (e.g., LevelDB, RocksDB, ScyllaDB) and distributed systems on top 
of them (e.g., Bigtable~\cite{Chang2008}, Cassandra\footnote{\small{\url{http://cassandra.apache.org/}}}, HBase, 
MongoDB\footnote{\small{\url{http://mongorocks.org/}}}, MySql\footnote{\small{\url{http://myrocks.io/}}}), etc. 
%which are employed by  Apache HBase, Bigtable, RocksDB, LevelDB, MongoDB, SQLite4, WiredTiger, Apache Cassandra,  InfluxDB, and many more.
The premise for using LSM trees is that disk access is considered the principal bottleneck in storage systems, even with today's SSD hardware~\cite{rocksdb,Tanenbaum:2014:MOS:2655363,Wu:2012:AWB:2093139.2093140}. 
The main design goal is to improve write throughput, which LSM trees do by batching writes in memory 
and periodically \emph{flushing} the memory  to disk. This way, random-access I/O is transformed to storage-friendly sequential I/O. 

An LSM tree includes a \emph{memory store} in addition to the large \emph{disk store} consisting of a collection of files. 
The memory store absorbs writes, and is periodically flushed to disk as a new ordered file. Reads search for the data
in the memory store, as well as in the disk store. The number of files has adverse impact on read performance. 
In order to reduce it, the system peridically runs a background \emph{compaction} process, which reads some files from 
the disk and merges them into a single file while removing redundant (overwritten or deleted) entries.%. For further background, see Section~\ref{sec:background}.

% Compactions hurt 
The performance of modern LSM stores is highly optimized, and yet as technology scales and real-time 
performance expectations increase, these systems face more stringent performance demands. In particular, 
they are extremely sensitive to the rate and extent of compactions. If compactions are infrequent, read performance
suffers (e.g., caching is less efficient when the data is scattered across multiple files). On the other hand, if 
compactions are too frequent, they jeopardize the performance of both writes and reads by consuming CPU 
and I/O resources. In formal terms, LSM trees struggle to balance between {\em read amplification} (the number 
of disk reads per query) and {\em write amplification} (the ratio of bytes written to disk versus bytes written to the 
database). In addition to its performance impact, write amplification accelerates device wearout, especially for SSD 
hardware~\cite{Hu:2009}. 


 
%One of the pain points with LSM stores is compactions \Idit{Can we cite something from RocksDB or another KV store here?}.
In HBase, a major pain point is the occurrance of so-called \emph{major compactions}, which replace all the files of a given table.
Indeed, HBase manuals and tuning tips are often aimed at reducing or even avoiding major compactions altogether.
 %\footnote{\url{https://blogs.msdn.microsoft.com/ashish/2016/09/02}}.

% Memory management
LSM trees can benefit from a large memory store, which absorbs many writes before each disk flush, 
and in turn delays the need for disk compactions, \Idit{Citations for this?} including major ones. 
The memory store is \emph{dynamic}, i.e., constantly changes, and must support fast lookup by key. 
It is therefore typically organized as a skip list
\Idit{Citations for this?} or a similar data structure.  Unfortunately, managing a large dynamic data structure induces 
non-negligble  overhead. 
Moreover, 
LSM-based data stores typically perform compactions only on disk and do not attempt to eliminate redundancies in the memory store.
Given that data access patterns often follow heavy-tailed distributions like Zipf,  this means that many versions of popular objects are 
wastefully kept in the in-memory data structure. 
Our work is motivated by this management overhead and waste. 

% Drumroll 
In Section~\ref{sec:accordion}
we introduce \sys, a new memory management scheme for LSM trees.
The principles behind \sys\ are 
\begin{enumerate}
\item \emph{in-memory compactions} can free up memory space to absorb more writes before the
memory store needs to be flushed, thus reducing compactions and disk wear; and
\item \emph{flattening} dynamic data structures into static buffers can reduce memory management overhead.
\end{enumerate}

% Implementation
In Section~\ref{sec:hbase} we  discuss on our implementation of  \sys\ in HBase, which is committed  as the default path in HBase 2.0. 
\Idit{Say something interesting about this process?}
% Evaluation
In Section~\ref{sec:eval} we experiment with \sys\ in a range of  scenarios -- with SSD and HDD storage, with different numbers
of regions, and different workloads. Surprisingly, we see that in virtually all settings, disk I/O is \emph{not} the principal performance bottleneck,
but rather memory management overhead is more substantial. By flattening the memory store \sys\ reduces this overhead and improves overall performance by \inred{XX\% to XX\%}. In-memory compactions have minor impact on performance at best, but do reduce 
write amplification by \inred{XX\% to XX\%}, which is important for extending disk life.

\Idit{Need to think of a good summary sentence for here. }




 

\section{Background}\label{sec:bg}
%\input{bg}

\section{\sys} \label{sec:accordion}
\subsection{Overview}

\begin{figure*}
\caption{\bf{LSM architecture based on compacting memstore.}}
\label{fig:compacting}
\end{figure*}

\sys\/ introduces a {\em compacting\/} memstore to the LSM tree design framework. Contrast to the traditional memstore, 
which maintains all the RAM-resident data in a single monolithic data structure, \sys\/ manages the data as a sequence of 
{\em segments}, ordered by creation time. At all times, the last created segment, called {\em active}, is mutable; it absorbs 
the put operations. The rest of the segments are immutable. The get and scan operations retrieve data from all the segments 
simultaneously, similarly to traditional LSM tree read from multiple levels. Figure~\ref{fig:compacting} illustrates the architecture. 

Once the active segment grows to a $\rho$ fraction of the memstore size bound, an {\em in-memory flush} happens.
The segment becomes immutable. The system queues it up to the list of inactive segments, called {\em pipeline}, 
and creates a new active segment. The  memstore periodically shrinks the pipeline in the background, by applying 
one or more {\em in-memory compaction} mechanisms: 
\begin{enumerate}
\item {\em Flattening of intra-segment search indices.} Since pipelined segments are immutable, it is sufficient to maintain 
their indices as compact ordered arrays, rather than reference-hungry skiplists. An additional advantage of the flat layout 
is that the index can be relocated to off-heap memory (unmanaged by the JVM), which improves performance predictability 
through reduced GC jitter~\cite{alibabahbase}. 
\item {\em Merging multiple segment indices.} A single index covering multiple segment data is created. Redundant data 
versions are not eliminated, to avoid comparisons and physical data copy. 
\item  {\em Merging multiple segment data.} Extends the above with redundant data elimination. The created flat index 
contains no redundancies. In case the memstore manages its cell data storage internally, the surviving cells are relocated
to new slabs; otherwise, the redundant cells are simply de-referenced, and later garbage-collected.    
\end{enumerate} 
The choice of compaction mechanisms is guided by the policies described in Section~\ref{sec:policies}. 

When the region server flushes a compacting memstore to disk, it scans the data from a snapshot 
of the pipelined segments. Similarly to the traditional design, the scan eliminates the redundancies 
prior to flush. 
 
%Summing up, in contrast with traditional memstores that can only grow between disk flushes, compacting memstores 
%can both expand and contract, resembling the movement of accordion bellows. 
%In-memory compactions that reduce the number of immutable segments bound the tail read latencies, as most reads access 
%a few segments. 

\subsection{Compaction Policies}
\label{sec:policies}

We implement three in-memory compaction policies: 
\paragraph{\basic} (low-overhead). Once a segment becomes immutable, flatten its index. Once the pipeline size exceeds $s$, 
merge all segment indices into one.  
\paragraph{\eager} (high-overhead, high-reward under redundancy-heavy workloads). 
In addition to \basic\/ mechanisms, eliminates data redundancies.
\paragraph{\adp} (the best of all worlds). The heuristic adapts to the recent workload, and selects either \basic\/ 
or \eager\/ optimizations depending on the context. \inred{Eshcar, please add the details}. 

\subsection{Implementation Details}

A compacting memstore is comprised from the active segment and a double-ended queue (pipeline) of inactive segments. 
The pipeline is accessed by read API's (get and scan), as well as by backgound disk flushes and in-memory compactions. 
The latter two modify the pipeline, by adding/removing/replacing segments. These modifications happen infrequently. 

The pipeline readers and writers coordinate through a lightweight copy-on-write, as follows. The pipeline object is versioned. 
Each modification promotes the version number, and atomically swaps the global reference to the new version clone. Note
that cloning is inexpensive -- only the segment references are copied since the segments themselves are immutable. 

The reads access the segments lock-free, through the version obtained at the beginning of the operation. If a disk flush
is scheduled in the middle of a read, a segment may migrate from the pipeline to the pre-flush snapshot buffer. The read 
safety is guaranteed by scanning the pipeline clone first. This way, a segment may be encountered twice but no data is 
lost. The scan algorithm filters out the duplicates. 

In-memory compaction is a read-modify-write operation, which swaps one or more segments in the pipeline 
with a new segment built from their data. This operation's atomicity is guaranteed by compare-and-swap (CAS), 
which flips pipeline version only if the latter did not change since the compaction started. Note that it is acceptable
for in-memory compaction to fail sometimes because it is an optimization that may be retried later. 

 

\section{Deployment in HBase} \label{sec:hbase}

\section{Empirical study} \label{sec:eval}

\section{Conclusion} \label{sec:conclusions}


%\subsection*{Acknowledgments}

\newpage

\bibliographystyle{acm}

\bibliography{refs}

\end{document}
